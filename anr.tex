\section{Análisis de los resultados}

Observando la tabla \ref{tab:Discrepancias} se nota que la
suma se hacerca mucho a cero lo que nos indica que el método
numérico se asemeja bastante al movimiento cuyos valores
se encontraron experimentalmente.

Observando el valor de la aceleración se observa que 
todos los conos alcanzan la velocidad terminal en tiempos
distintos. Mientras mayor es su peso más tarda en alcanzar 
su velocidad terminal.

Por último, comparando las gráficas de los valores obtenidos 
con el metodo numerico con los experimentales se nota
que la diferencia entre los valores va aumentando conforme
aumenta el tiempo.
