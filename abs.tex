% comentario sin sentido
% segundo sin sentido
\begin{center}
    {\huge Caída de conos}\\
    {\normalsize David Amaro Alcalá}\\
    {\normalsize Laboratorio 2}\\
    dav1494@ciencias.unam.mx\\
\end{center}

\section*{Resumen}

Utilizando el metodo numerico del metodo 
de Euler de medio paso sobre el movimiento
de un cono virtual y utilizando los restultados
de una practica anterior, se obtuvieron los
valores de su velocidad terminal.

\begin{Tabla}
    \centering
    \begin{tabular}{|c|c|}
        \hline
        \rowcolor{azulito} Cono & Velocidad terminal (m/s) \\
        \hline Cono 1 & 1.27234 \\
        \hline Cono 2 & 1.4610 \\
        \hline Cono 3 & 1.6243 \\
        \hline
    \end{tabular}
\end{Tabla}
