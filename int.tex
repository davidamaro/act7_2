\section{Introducción}

Aunque en la mayoría de casos estudiados se desprecia
la fuerza de resistencia del aire, pero a grandes velocidades
su intervención en el movimiento de un objeto es importante tomarla 
en cuenta.

La ecuación en caída libre de un objeto con resistencia es:

\begin{equation}
    F = W + f = -mg + rv^2
    \label{int_caidalibre}
\end{equation}

donde $m$ es la masa del objeto cayendo, $g$ la aceleración de la
gravedad, $r$ el coeficiente de fricción del fluido sobre el que se esta
cayendo y $v$ la velocidad del objeto.

En caída libre el objeto cayendo alcanza la llamada \textit{velocidad terminal}
debido a la fuerza de resistencia del aire.

Como es fácil ver, el cono alcanza esta velocidad $v_t$ cuando $a$ es cero, es decir
\[
    -mg + rv^2 = 0,
\]
desarrollando:
\[
    mg = rv^2_t
\]

$r$ se puede calcular con la siguiente expresión:

\begin{equation}
    r = \frac{\rho A C_D}{2}
    \label{int_calcoef}
\end{equation}

donde $\rho$ es la densidad del aire, $A$ es el área que tiene
contacto con el fluido y $C_D$ es un coeficiente de arrastre
que depende de la forma del objeto cayendo.

Despejando a $C_D$ obtenemos

\begin{equation}
    C_D = \frac{2r}{\rho A}
    \label{int_valorDeR}
\end{equation}

Que nos permitirá conocer este último coeficiente después de determinar
$r$.

Finalmente para concer el área de la superficie de contacto 
de los conos desarrollamos lo siguiente.

\begin{equation}
    A_c = \pi R g
    \label{AreaCono}
\end{equation}

donde $A_c$ es el área del cono, $R$ el radio del círculo.

También sabemos que:

\begin{equation}
    P = 2 \pi r
    \label{Perimetro}
\end{equation}

donde $P$ es el perímetro del cono y $r$
es el radio del círculo que formará al cono.

Otra expresión util es:

\begin{equation}
    S = r \theta
    \label{Arco}
\end{equation}

con $S$ el arco del círculo y $\theta$ el ángulo
de la parte que se mutilo al círculo base.

\subsection{Metodo de medio paso}

El método de medio paso necesita de la poscición, la velocidad
promedio de las dos velocidades al principio y al final del 
intervalo de tiempo, esto es:

$v_{k+1} = v_k + \Delta t a_k$ y $y_{k+1} = y_k + \Delta \frac{v_{k+1}+v_k}{2}$

Uniendo ambas ecuaciones obtenemos que

$y_k = y_k + \Delta t v_k + \frac{1}{2}a_k \Delta t^2$

Nosotros utilizamos una variación del método anterior, en la
cual se comienza con un intervalo de tiempo $\Delta t/2$ en
lugar del $\Delta t$. De esta manera se obtiene

$v_{k+1/2} = v_k + \frac{\Delta t}{2}a_k$

A partir de este cálculo, el método sigue aplicándose
en intervalos de $\Delta t$. Así, la iteración continúa para
$k \neq 0$, como
$v_{k+1/2} = v_{k-1/2} + \Delta ta(v_k)$
$y_{k+i} = y_k + \Delta t v_{k+1/2}$
$v_(v_{k+1/2})$
